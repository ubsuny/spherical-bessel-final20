\documentclass[letterpaper,12pt]{article}
\usepackage[utf8]{inputenc}
%\renewcommand{\familydefault}{\sfdefault}
\usepackage{amsmath, amsthm, amssymb, amsfonts, IEEEtrantools, graphicx, multicol, caption, float, fancyhdr, lastpage, wrapfig, titlesec, authblk, subcaption, multirow, physics,stackengine,tensor}
\usepackage[dvipsnames,table]{xcolor}
\usepackage[makeroom]{cancel}
\setlength{\arrayrulewidth}{.5mm}
\setlength\parindent{0pt}
%\graphicspath{ {Figures/} }
%\numberwithin{equation}{section}
%\numberwithin{figure}{section}
%\numberwithin{table}{section}
%\setcounter{section}{11}
\usepackage[letterpaper, total={6.5in, 7.5in}]{geometry}
\usepackage[backend=biber,style=alphabetic, sorting=nyt]{biblatex}
\addbibresource{Sources.bib}


%Give PDF Links
%______________________________________________________________________
\usepackage[bookmarks]{hyperref}
%\texorpdfstring{TEX text}{Bookmark Text}
%\href{url}{text}
\hypersetup{colorlinks=false}
%______________________________________________________________________

%Custom Math Commands
%______________________________________________________________________
%Looks
\newcommand{\mc}[1]{\mathcal{#1}}
\newcommand{\mb}[1]{\mathbb{#1}}
\newcommand{\hlw}[1]{\hspace{#1\linewidth}}

%Equations
\newcommand{\eq}[1]{\begin{equation}#1\end{equation}}
\newcommand{\eqb}[1]{\begin{equation}\boxed{#1}\end{equation}}
\newcommand{\eqq}[1]{\begin{equation*}#1\end{equation*}}
\newcommand{\eqqb}[1]{\begin{equation*}\boxed{#1}\end{equation*}}
\newcommand{\ea}[1]{\begin{align}#1\end{align}}
\newcommand{\eaa}[1]{\begin{align*}#1\end{align*}}
%______________________________________________________________________

\pagestyle{fancy}
\fancyhf{}
\lhead{Eigenfrequencies of a Spherical Electromagnetic Cavity}
\rhead{Alexander Bivolcic}
\rfoot{Page \thepage \hspace{3pt} of \pageref{LastPage}}

%\makeatletter
%\@addtoreset{section}{part}
%\makeatother
\renewcommand\thepart{\arabic{part}}

\titleformat
 {\section}
 {\centering\color{PineGreen} \bfseries\Large}
 {\textcolor{PineGreen}{\thesection{}.}}
 {4pt}
 {}
 [\textcolor{PineGreen}{\rule{\textwidth}{0.3pt}}]
\renewcommand\thesection{\Roman{section}}

\titleformat
 {\subsection}
 {\centering\color{PineGreen} \bfseries\large}
 {\textcolor{PineGreen}{\thesubsection{}:}}
 {4pt}
 {}
\renewcommand\thesubsection{\Alph{subsection}} 

\titleformat
 {\subsubsection}
 {\centering\color{PineGreen} \itshape}
 {\textcolor{PineGreen}{\thesubsubsection{}:}}
 {4pt}
 {}
\renewcommand\thesubsubsection{\arabic{subsubsection}}


\title{ \normalfont\normalsize
\textcolor{PineGreen}{\rule{\linewidth}{0.5pt}}\\
\vspace{17pt}
\textcolor{PineGreen}{{\huge Eigenfrequencies of a Spherical Electromagnetic Cavity}}\\
\vspace{9pt}
\textcolor{PineGreen}{\rule{\linewidth}{2pt}}\\
}

\author{
{\Large Alexander J. Bivolcic}\\ 
{\small \textit{University at Buffalo}}\\
{\small\texttt{ajbivolc@buffalo.edu}}
}

\date{\normalsize\today}


\begin{document}

\begin{titlepage}
\maketitle
\begin{abstract}
    The following is the write up for my computational physics final on the topic of \emph{Eigenfrequencies of a Spherical Electromagnetic Cavity}. This pdf writeup stands in for the wiki which comes with github as it is unable to render latex. An explanation for how this file was compiled can be found both in the readme and the wiki of the repository and the latex source files can also be found in the repository.
\end{abstract}
\thispagestyle{empty}
\end{titlepage}

\tableofcontents
\section{Physics Background}

\subsection{Resonant Cavities}
We begin our discussion with a description of the physics problem at hand, as well as a derivation for the mathematics used. Implicitly in the title of this project we are looking at \emph{resonant cavities}, which are volumes used to store standing waves. In the case of Electromagnetic waves that means that the walls of this cavity are perfect conductors. Most physics students will be able to draw parallels to a vibrating string as it is a common wave PDE problem. A one dimensional parallel for electromagnetic waves would be two parallel mirrors separated by a distance L. In this case, the normal modes are electromagnetic waves that bounce between the mirrors such that the total trip is an integer number over wavelengths\cite{Zangwill}. This would be analogous to the \emph{standing wave} on the string where the nodes and antinodes are stationary.

The approach we will use to find the standing waves of a conducting cavity focuses on time-harmonic and divergence-free solutions of the homogeneous wave equation in a volume $V$ with surface $S$\footnote{At this point I am showing my future aspirations of being a theorist by using natural units $c=\hbar=1$. This should make little difference to the physics we are studying, and it makes the equations slightly nicer.}.
\ea{
\qty(\laplacian + \epsilon\mu\omega^2)\va{E} &= 0 \ \text{ in V} \label{eq:Helmholtz}\\
\div \va{E} &= 0 \ \text{ in V}\\
\value{n} \cross \va{E} &= 0 \ \text{ on S}
}
Here $\epsilon$ is the electric permutivity of the volume, $\mu$ is the magnetic permutivity, and $\va{E}$ is the electric field which has the following spacial and temporal components:
\eq{
\va{E} = \va{E}(\va{r})e^{-i\omega t}
}
Solving the Helmholtz equation \eqref{eq:Helmholtz} is analytically possible for simple geometries using separation of variables. These geometries include variations on cylinders, rectangular prisms, and spheres. For further reading on the solutions to the Helmholtz equation you are welcome to suffer like every other Physics graduate student and read \cite{Jackson}.

\subsection{Derivation for a Spherical Cavity}
Armed with a deeper understanding of resonant cavities, we can now turn our attention to the problem at hand, spherical cavities. The electromagnetic normal modes of a spherical resonant cavity are time-harmonic, vector spherical waves that satisfy the perfect-conductor boundary condition $\vu{r}\cross \va{E}\big{|}_s=0$ at the cavity’s walls \cite{Zangwill}. If we take $u(r,\theta,\phi)$ to be the solution to \eqref{eq:Helmholtz} then our normal modes would take the form of transverse magnetic and transverse electric waves.
\ea{
\va{E}_E = -i\omega\va{r}\cross\grad u &\hlw{.05} \va{E}_M = \curl(\va{r}\cross\grad u)\\
\va{B}_E = -\curl(\va{r}\cross \grad u) &\hlw{.05} \va{B}_M = -i\omega\va{r}\cross\grad u
}
Most readers will find a parallel to other areas of Electrodynamics in that our solution to the Helmholtz equation will be a linear combinations of radial equations multiplying the spherical harmonics. In this case our radial function will be a combination of spherical Bessel functions ($j_l(kr)$) and spherical Neumann functions ($n_l(kr)$.
\eq{
u(\va{r}) = \sum_{lm}\qty(A_l j_l(kr) + B_l n_l(kr))Y_{lm}(\theta,\phi)
}
Here $k=\frac{\omega}{c}$. It will be related to the zeros of these Bessel functions, which is how we will obtain our Eigenvalues or Eigenfrequencies. In this case we are only concerned with behavior inside the cavity. Since $n_l$ diverges are the origin we are able to force all $B_l$ to be zero\cite{Jackson}. Thus the form of $u$ inside the cavity is the following:
\eqb{
u_{lm}(\va{r}) = A_l j_l(kr))Y_{lm}(\theta,\phi)
}
Computing the various transverse waves above will require some vector calculations on this solution. The following identity will prove to be quite useful due to the nature of spherical harmonics.
\eq{
\curl(\va{r}\cross\grad u) = \va{r}\laplacian u - 2\grad u - r\pdv{r}\grad u
}
This allows us to write the form of the transverse electric and magnetic waves\cite{Zangwill}.
\ea{
E_E &= i\omega j_l(kr)\qty(\frac{1}{\sin\theta}\pdv{Y_{lm}}{\phi}\vu{\theta}-\pdv{Y_{lm}}{\theta}\vu{\phi}) \label{eq:wave1}\\
\nonumber\\
E_M &= -\qty(\frac{l(l+1)}{r^2}\va{r}+\qty[\vu{\theta}\pdv{\theta}+\vu{\phi}\frac{1}{\sin\theta}\pdv{\phi}]\frac{1}{r}\pdv{r})rj_l(kr)Y_{lm}(\theta,\phi) \label{eq:wave2}
}

\subsection{Analytic Solutions}

In order to make sure that we have a complete understanding of the Bessel functions we will do a slight review. Bessel functions are the solutions to Bessel's differential equation:
\eq{
x^2\dv[2]{y}{x} + x\dv{y}{x}+(x^2-m^2)y = 0
}
Most readers will be familiar with the form of cylindrical Bessel functions as the solution to the Laplace equation in cylindrical coordinates. These functions occur in the case where $m$ is an integer value. The solutions to the Helmholtz equation are the spherical Bessel functions. These functions occur when $m$ is a half integer. In such a case we make the transformation $m^2\to l(l+1)$ so that $l$ can be used as an integer index for the functions. It is customary to represent spherical Bessel functions will lowercase letters and cylindrical Bessel functions with uppercase\cite{handbook}. While these functions can be related to cylindrical Bessel functions, they are unique with respect to the fact that there is a much simpler way of writing them. Specifically we will use Rayleigh's formulas and then push forward.
\ea{
j_l(x) &= (-1)^l \qty(\dv{x})^l\frac{\sin(x)}{x}\\
n_l(x) &= (-1)^l+1 \qty(\dv{x})^l\frac{\cos(x)}{x}\\
}
While we could go down an endless rabbit hole when it comes to Bessel functions, I believe this will be enough for the problem at hand.

We're most interested in finding the Eigenfrequencies of the spherical cavity. To do so we look back at the boundary condition $\vu{r}\cross \va{E}\big{|}_s=0$. When applied to equations \eqref{eq:wave1} and \eqref{eq:wave2} this means that the angular comonents of those waves must be zero for any value all values of $\theta$ and $\phi$ when $r=R$ (the outer radius of the vanity). The only way to do so is to set either the Bessel function of the derivative of the Bessel function to zero at the walls.
\ea{
j_l(k_{n,l}R) &= 0\\
\dv{r}\qty(rj_l(k_{n,l}r))\big{|}_{r=R} &= 0 \label{eq:annoying}
}
%The magnetic wave equation requires us to make use of the derivative relation
%\eq{\qty(\frac{1}{z}\dv{z})^m \qty(z^{l+1}J_l(z)) = z^{l-m+1}J_{l-m}(z)}
%Thus simplifying equation \eqref{eq:annoying} to the following:
%\eq{\dv{r}\qty(rj_l(kr)) = \frac{1}{k}\dv{kr}\qty(rj_l(kr))}

This allows us to write down the Eigenfrequencies in terms of the zeros of the bessel functions ($x_n,l)$.
\eqb{
\omega = x_{n,l}\frac{c}{R}
}
The zeros have multiple frequencies as the zeros of the $l$ bessel function are not the same as the zeros of the $l+1$ function. Moroever, due to the periodic nature of $\sin$ and $\cos$ each order of Bessel function has many zeros.

At this point it is clear that our goal is calculate the zeros of the various Bessel functions. Doing so by hand is rather straight forward, but it takes time, especially for higher order Bessel functions. Since these functions have been around for so long most people consult a table of zeros when they are needed. In our case

\section{Computational Solutions}
%Be sure to mention what you are using and why
Here is some stuff


\section{Applications and Current Research}

\section{Conclusion}
We began by looking at the conditions for a Spherical Electromagnetic Cavity. We were able to successfully use separation of variables to solve the Helmholtz equation is spherical coordinates in order to find the Eigenfunctions for this specific geometry. This allowed us to relate the zeros of Spherical Bessel functions, as well as the size of the cavity to the Eigenfrequencies of these resonant cavities. We then looked at the various analytic ways to compute these Bessel functions and some computational methods implemented by others. Finally we were able to devise a strategy more efficiently find the zeros of higher Eigenfrequencies using the trig functions that make up the Spherical Bessel Functions. 

\printbibliography[title={References}]


\end{document}